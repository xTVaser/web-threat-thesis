\chapter{Introduction}

Micro-optimization is a technique to reduce the overall operation count of
floating point operations.  In a standard floating point unit, floating
point operations are fairly high level, such as ``multiply'' and ``add'';
in a micro floating point unit ($\mu$FPU), these have been broken down into
their constituent low-level floating point operations on the mantissas and
exponents of the floating point numbers.

Chapter two describes the architecture of the $\mu$FPU unit, and the
motivations for the design decisions made.

Chapter three describes the design of the compiler, as well as how the
optimizations discussed in section~\ref{ch1:opts} were implemented.

Chapter four describes the purpose of test code that was compiled, and which
statistics were gathered by running it through the simulator.  The purpose
is to measure what effect the micro-optimizations had, compared to
unoptimized code.  Possible future expansions to the project are also
discussed.


\begin{figure}
\vspace{2.4in}
\caption{Armadillo slaying lawyer.}
\label{arm:fig1}
\end{figure}

\begin{table}
\vspace{2.4in}
\caption{Armadillo slaying lawyer.}
\label{arm:fig1}
\end{table}

\section{Problem Definition}

The idea of micro-optimization is motivated by the recent trends in computer
architecture towards low-level parallelism and small, pipelineable
instruction sets \cite{small}.  By getting rid of more
complex instructions and concentrating on optimizing frequently used
instructions, substantial increases in performance were realized.

Another important motivation was the trend towards placing more of the
burden of performance on the compiler.  Many of the new architectures depend
on an intelligent, optimizing compiler in order to realize anywhere near
their peak performance.  In these cases, the
compiler not only is responsible for faithfully generating native code to
match the source language, but also must be aware of instruction latencies,
delayed branches, pipeline stages, and a multitude of other factors in order
to generate fast code.

Taking these ideas one step further, it seems that the floating point
operations that are normally single, large instructions can be further broken
down into smaller, simpler, faster instructions, with more control in the
compiler and less in the hardware.  This is the idea behind a
micro-optimizing FPU; break the floating point instructions down into their
basic components and use a small, fast implementation, with a large part of
the burden of hardware allocation and optimization shifted towards
compile-time.

Along with the hardware speedups possible by using a $\mu$FPU, there are
also optimizations that the compiler can perform on the code that is
generated.  In a normal sequence of floating point operations, there are
many hidden redundancies that can be eliminated by allowing the compiler to
control the floating point operations down to their lowest level.  These
optimizations are described in detail in section~\ref{ch1:opts}.

\section{Objective}\label{ch1:opts}

In order to perform a sequence of floating point operations, a normal FPU
performs many redundant internal shifts and normalizations in the process of
performing a sequence of operations.  However, if a compiler can
decompose the floating point operations it needs down to the lowest level,
it then can optimize away many of these redundant operations.  

If there is some additional hardware support specifically for
micro-optimization, there are additional optimizations that can be
performed.  This hardware support entails extra ``guard bits'' on the
standard floating point formats, to allow several unnormalized operations to
be performed in a row without the loss information\footnote{A description of
the floating point format used is shown in figures
and. A discussion of the mathematics behind
unnormalized arithmetic is in appendix.}

\section{Thesis Overview}
