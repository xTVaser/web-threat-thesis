\chapter{Current Detection and Prevention Methods}

\section{Prevention through Development}

These web attacks can be significantly damaging to an organization in many ways and so detecting and preventing them is of a very high importance.  These problems are also not limited to only small time websites as well, with many high profile websites already being victim to attacks, and some studies stating that over 90\% of web applications being vulnerable to SQL injections alone. % detection and prevention of SQL injections

The best way to stop these attacks is to prevent the vulnerabilities from existing in the first place on the development side.  Despite the fact that majority of the attacks are well documented and understood, as safeguards are preventions are put in place to protect one aspect of an application, attacks shift their efforts to look for the next weakest link.  From a security standpoint you need to assume that your application is not bulletproof and that you have only mitigated the risk, not removed it completely.  As a result, prevention alone is not enough as there is always the possibility that an attack finds a way past the safe guards and so prevention and detection must work hand in hand.

The simpliest and most common way to prevent these attacks that are at the application layer level of the web application is to never allow user input to be directly concatenated into any command that interacts on the server side.  This is done by making use of what is called prepared statements, you instead construct the entire query or command and then pass in your data from the user as parameters.  This allows the server to distinguish between data and code no matter what kind of input is supplied by the user.  Likewise, whenever data is accessed from storage to be displayed to the user, it should not be directly inserted into a command that could potentially treat it as arbitrary code.  This will prevent issues like stored XSS attacks or RFIs from occuring where stored code is inserted into the flow of the application code.  % OWASP sql injection prevention 
Many languages have different ways of accomplishing this but the most simpliest way is to simply strip the portions that cause it to be interpreted as code, but a more comprehensive way is to filter the HTML against a whitelist. % html purifier

There are other measures that can be taken but these mostly pertain to the environment on which the application is ran on and specifically for SQL injection prevention.  Seperate database users should be made for each application and they should have the least amount of privledge possible.  In the event that something is compromised, that database is at least isolated and other applications are uneffected.  A second measure that can be taken is to use views extensively instead of using direct queries for all database interactions as it allows access to the tables to be denied and only to the specially tailored views.  These strategies embrace the least privledge idea of security, where it is an unnessecary risk to be privy to more information or have more access than you need % owasp sql injection prevention

Research has been done on examing the code of potentially XSS vulnerable applications to determine their vulnerability and was able to accurately detect the vulnerable code with no false positives or negatives.  So it is clear that modifying the code itself to be more safe should always be the first and most important step if it is possible to determine if a particular file is vulnerable that easily, detection is a nessecary backup plan. %XSSDM

\section{Signature Based Detection}

A very traditional way of detecting for security threats is the use of signatures, however many of these signature based tools are more suitable for the lower levels of the OSI model rather than the application and presentation levels. %cite main paper
These tools are referred to as Intrusion Detection Systems (IDS) and rely on regular expressions and other pattern matching tools produced using existing or previous attacks, one example of such a tool is Snort. %cite snort
Therefore, as long as there is an adequate number of signatures that cover the broad spectrum of possible attacks then the technique can be quite accurate.  

However signature based detection systems as well as other IDS systems can have many problems associated with them.  One of the biggest problems is the frequency of false positives, when the system believes that something that is not harmful is.  Of course the opposite is also true and IDS systems can let attacks slip by, this can be caused by the attacks using various tricks to evade detection such as using alternate encodings or fragmenting packets. % on the verification
In addition, if the IDS is signature based than what is most likely the problem for these accuracy problems is a lack of signatures that are either more accurate or cover new undocumented attacks.  It is becoming much too impractical to produce these signatures fast enough due to the countless variants of the attacks and that the attacks are commonly designed to be targetted and go unnoticed rather than spread as fast as possible like with conventional computer attacks. %trend micro whitepaper
To give an idea on how difficult of a problem this is to solve with signatures, an average of 5,000 new software vulnerabilities have been identified per year; and with the number of unique malware programs alledgely in the tens of millions and doubling every year.  With these rising trends, it is clear that the malicious user is easily always ahead of the detection tools, static solutions such as signature sets are becoming less and less practical every year, and the current short comings of the detection sysytems themselves proves that point.  % on the verification

However, this problem is not unique to just the web threat world, although signature based detection is much more suited for the traditional desktop computer application virus scanning practices have had to adapt as well to a similar problem.  Virus scanning is probably the best example of signature based detection in action, where malware is collected and a signature is developed to detect it and then sent out to the masses as fast as possible.  However, some types of viruses have begun to exploit this by transforming their own code when transferring which would require an entire new signature to detect.  These so called Metamorphic viruses are not impossible to defeat but they require approaching the idea of scanning for a virus completely differently than just collecting a signature, such techniques include but are not limited to hidden Markov models or reversing the morphing process of the malware. % are metamorphic, hunting for undetectable
If the area where signature based detection is the most strong has to adapt tactics to deal with the changing environment, then by extension so to it does the web threat detection ecosystem.

\section{Modern Methods of Detection}

In order to combat these challenges for web threats, some people have suggested that a multiple layered approach will provide the best defense.  Such a system would not only have multiple layers of detection but also feedback loops to process the information and update the protection systems for future detection.  A multi-layered approach would also be able to address all levels of the network rather than a system for only the network layers, and another for the application layers.  Such an approach would also enable for portions of the processing to be centralized and on the cloud while other areas be closer to the endpoint.  Traditional techniques like signature detection would still be used, but it would be able to be augmented with behaviour analysis for example as often times web attacks are carried out in massive enumeration attempts and not a single bad request.  One final point is that such a solution would allow for global collaboration to contribute to reputation lists, whitelists, and the like to further solve the problem of a growing threat instead of having the same tools deployed in multiple areas and not constantly and consistently updated. % trend micro

This kind of a multi-layered approach combines the best of the old techniques with new potential solutions and most interestingly suggests a system that is inherently evolutionary, growing and improving as a core trait. 








